\documentclass{beamer}
\usepackage{listings}
\usepackage{color}
\usepackage{subfig}
\usepackage{wrapfig}

%FIRST FRAME
\usetheme{Luebeck}
\usecolortheme{crane}
\setbeamertemplate{blocks}[rounded][shadow=true]
\title{IL TIEPIDO GHIACCIO DEL MONTE FUJI}
\subtitle{Quel che resta, resterà?}
\author{Ludovico Chieffallo}
\date{\nodate}
\institute{Alma Mater Studiorum, Università di Bologna\\
\bigskip
\includegraphics[width=0.5\textwidth]{unibo.png}
}



\begin{document}
\maketitle
\AtBeginSection[]
{
\begin{frame}{Indice}
\tableofcontents[currentsection,currentsection,currentsectio]

\end{frame}
}


%%%%%%%%%%%%%%%%%%%%%%%%%%%%%%%%%%%%%%%%%%%%%%%%%%%%%%%%%%%%%%%%%%%%%%
%%%%%%%%%%%%%%%%%%%%%%%%%%%%%%%%%%%%%%%%%%%%%%%%%%%%%%%%%%%%%%%%%%%%%%
\section{Introduzione}
\subsection{Cosa sta succedendo?}
\begin{frame}{Cosa sta succedendo?}\pause
\begin{itemize}
    
\item La NASA ha rilevato tramite satelliti, che il manto nevoso del Monte Fuji, nel dicembre 2021 ha registrato l'indice più basso degli ultimi 20 anni.\pause

\item Le stazioni metereologiche hanno confermato le stesse osservazioni.\pause

\item Lo scienziato Toshio Iguchi ha rilevato precipitazioni molto inferiori rispetto al solito.\pause

\item È stato rilevato che il limite del bosco della montagna si sia spostato verso l'alto di 30 metri negli ultimi 40 anni, ciò è stato attribuito all'innalzamento delle temperature di circa 2 gradi Celsius.
\end{itemize}
\end{frame}

%%%%%%%%%%%%%%%%%%%%%%%%%%%%%%%%%%%%%%%%%%%%%%%%%%%%%%%%%%%%%%%%%%%%%%%
\subsection{Qual è lo scopo del progetto?}

\begin{frame}{Qual è lo scopo del progetto?}\pause

\begin{tabular}{cl}  
    \begin{tabular}{l}
        \includegraphics[height=6cm, width=5.1cm]{introduzione.jpg}
    \end{tabular}
    & \begin{tabular}{5}
        \parbox{0.4\linewidth}{%  change the parbox width as appropiate
            \begin{itemize}
               \item Indagare il grado di questi cambiamenti.\pause
            
                    \item Analizzare le cause.\pause
            
                \item Analizzare i possibili scenari Futuri 

            \end{itemize}
        }
    \end{tabular}  \\
\end{tabular}
\end{frame}

%%%%%%%%%%%%%%%%%%%%%%%%%%%%%%%%%%%%%%%%%%%%%%%%%%%%%%%%%%%%%%%%%%

\begin{frame}{Monte Fuji, 1 Gennaio 2021}

\centering
\includegraphics[width=0.9\textwidth]{introduzione2.jpg}
    
\end{frame}

%%%%%%%%%%%%%%%%%%%%%%%%%%%%%%%%%%%%%%%%%%%%%%%%%%%%%%%%%%%%%%%%%%

\subsection{Come sono state scaricate le immagini?}

\begin{frame}{Come sono state scaricate le immagini?}
Sono state utilizzate varie fonti per scaricare le immagini di questo progetto:
\bigskip
\begin{itemize}
    \item NASA Earth Observatory
    \item GloVis USGS
    \item Google Earth Engine
    \end{itemize}

    
\end{frame}

%%%%%%%%%%%%%%%%%%%%%%%%%%%%%%%%%%%%%%%%%%%%%%%%%%%%%%%%%%%%%%%%%%%%
%%%%%%%%%%%%%%%%%%%%%%%%%%%%%%%%%%%%%%%%%%%%%%%%%%%%%%%%%%%%%%%%%%%%


\section{Classificazione}
\subsection{Pacchetti di programmazione(R)}
\begin{frame}{Pacchetti di programmazione (R)}
Per questo progetto sono stati utilizzati vari pacchetti di programmazione
\scriptsize \lstinputlisting[language=R]{prova.txt}
\end{frame}

%%%%%%%%%%%%%%%%%%%%%%%%%%%%%%%%%%%%%%%%%%%%%%%%%%%%%%%%%%%%%%%%%%%
\subsection{Raster e  RStoolbox}
\begin{frame}{Raster e  RStoolbox}
Per la classificazione sono stati usati i pacchetti "raster" e "RStoolbox"
\lstinputlisting[language=R]{rs.txt}

\end{frame}

%%%%%%%%%%%%%%%%%%%%%%%%%%%%%%%%%%%%%%%%%%%%%%%%%%%%%%%%%%%%%%%%%%%

\begin{frame}{Importazione file e analisi}

A questo punto si è potuti passare all'importazione dei dati e alla classificazione:
\bigskip

\scriptsize\lstinputlisting[language=R]{un.txt}
    
\end{frame}

%%%%%%%%%%%%%%%%%%%%%%%%%%%%%%%%%%%%%%%%%%%%%%%%%%%%%%%%%%%%%%%%%%%%%

\begin{frame}{PlotRGB}
\centering
\includegraphics[width=1\textwidth]{Rplot.png}
    
\end{frame}

%%%%%%%%%%%%%%%%%%%%%%%%%%%%%%%%%%%%%%%%%%%%%%%%%%%%%%%%%%%%%%%%%%%%
%%%%%%%%%%%%%%%%%%%%%%%%%%%%%%%%%%%%%%%%%%%%%%%%%%%%%%%%%%%%%%%%%%%%
\section{Analisi multitemporale}

\subsection{Codice}

\begin{frame}{Analisi Multitemporale}
A questo punto possiamo vedere visivamente quanto è cambiata la copertura di neve nell'area del monte Fuji:
\bigskip
\scriptsize\lstinputlisting[language=R]{name.txt}
    
\end{frame}


%%%%%%%%%%%%%%%%%%%%%%%%%%%%%%%%%%%%%%%%%%%%%%%%%%%%%%%%%%%%%%%%%%%

\begin{frame}{Plot della differenza di neve 2021/2013}
Qui possiamo vedere il plot risultante:
\bigskip
    \centering
    \includegraphics[width=0.7\textwidth]{Rplot02.png}   

\end{frame}

%%%%%%%%%%%%%%%%%%%%%%%%%%%%%%%%%%%%%%%%%%%%%%%%%%%%%%%%%%%%%%%%%%%%
\subsection{Possono vedere tutti queste mappe?}

\begin{frame}{Possono vedere tutti queste mappe?}\pause

Purtroppo non tutti possono vedere queste mappe.

Le persone affette da CVD (colour vision deficienty) non riescono a distinguere alcuni colori. 

\centering
 \includegraphics[width=0.9\textwidth]{cvd.jpg}   

\end{frame}

%%%%%%%%%%%%%%%%%%%%%%%%%%%%%%%%%%%%%%%%%%%%%%%%%%%%%%%%%%%%%%%%%%%%%

\begin{frame}{Frame Title}
    
Questo è un problema poichè rende le nostre mappe non scientificamente leggibili per tutti.

La soluzione è usare delle palette che permettano a tutti di comprendere queste mappe.

Useremo quindi la palette VIRIDIS facilmente comprensibile per tutti 

\end{frame}    


%%%%%%%%%%%%%%%%%%%%%%%%%%%%%%%%%%%%%%%%%%%%%%%%%%%%%%%%%%%%%%%%%%%
\begin{frame}{Plot della differenza di neve 2021/2013}
 
\includegraphics[width=1\textwidth]{Rplot01.png}   
    
\end{frame}

%%%%%%%%%%%%%%%%%%%%%%%%%%%%%%%%%%%%%%%%%%%%%%%%%%%%%%%%%%%%%%%%%%%%
\section{NDSI}

\begin{frame}{NDSI (Normalized Difference Snow Index)}
  Il Normalized Difference Snow Index (NDSI) è un indice correlato alla presenza della neve.
  
  La neve ha tipicamente una riflettanza visibile (VIS) molto elevata e una riflettanza molto bassa nell'infrarosso a onde corte (SWIR).
  
  Questa è una caratteristica utilizzata per rilevare la neve distinguendola dalla maggior parte dei tipi di nuvole. Il manto nevoso viene rilevato utilizzando il rapporto NDSI.
  \bigskip
  
  Il rapporto NDSI consiste in questa differenza:
  \bigskip
  \centering
  \begin{equation}
NDSI=\frac{green-swir^1}{green+swir^1}
\nonumber
\end{equation}

\end{frame}


%%%%%%%%%%%%%%%%%%%%%%%%%%%%%%%%%%%%%%%%%%%%%%%%%%%%%%%%%%%%%%%%%%%%

\begin{frame}{Qual è la soglia per capire se la neve è presente??}
 \centering
 Si considera che un pixel con NDSI $>$ 0.0 abbia della neve presente. Un pixel con NDSI $<=$ 0.0 è una superficie terrestre priva di neve (Riggs et al., 2016).   
    
\end{frame}


%%%%%%%%%%%%%%%%%%%%%%%%%%%%%%%%%%%%%%%%%%%%%%%%%%%%%%%%%%%%%%%%%%%%
\subsection{Codice NDSI}
\begin{frame}{Importare le immagini}
Per prima cosa possiamo importare tutte le nostre immagini per creare la nostra immagine multispettrale:
\bigskip
\scriptsize\lstinputlisting[language=R]{ndsi.txt}
    
\end{frame}

%%%%%%%%%%%%%%%%%%%%%%%%%%%%%%%%%%%%%%%%%%%%%%%%%%%%%%%%%%%%%%%%%%%%
\begin{frame}{Crop delle immagini}
Poiche sono delle immagini satellitari (LANDSAT 8) molto estese, abbiamo bisogno di fare un crop sulla zona di interesse:
\bigskip
\scriptsize\lstinputlisting[language=R]{crop.txt}
    
\end{frame}

%%%%%%%%%%%%%%%%%%%%%%%%%%%%%%%%%%%%%%%%%%%%%%%%%%%%%%%%%%%%%%%%%%%%%%
\begin{frame}{PlotRGB con estensione sul monte fuji}
\centering
\includegraphics[width=1\textwidth]{Rplot03.png}  
    
\end{frame}

%%%%%%%%%%%%%%%%%%%%%%%%%%%%%%%%%%%%%%%%%%%%%%%%%%%%%%%%%%%%%%%%%%%%%%
\begin{frame}{Calcoliamo NDSI}
Sappiamo quindi che la nostra formula è:
  \bigskip
  \centering
  \begin{equation}
NDSI=\frac{green-swir^1}{green+swir^1}
\nonumber
\end{equation}
\bigskip
Ma a questo punto come sappiamo quale bande usare?


\end{frame}
%%%%%%%%%%%%%%%%%%%%%%%%%%%%%%%%%%%%%%%%%%%%%%%%%%%%%%%%%%%%%%%%%%%%%%

\begin{frame}{Calcoliamo NDSI}
\centering
Basta vedere la composizione di Landsat 8

\bigskip
\centering
\includegraphics[width=0.5\textwidth]{oniono skin landsat.JPG}  
\end{frame}

%%%%%%%%%%%%%%%%%%%%%%%%%%%%%%%%%%%%%%%%%%%%%%%%%%%%%%%%%%%%%%%%%%%%%%%
\begin{frame}
Finalmente possiamo calcolare NDSI
 \bigskip
 \scriptsize\lstinputlisting[language=R]{ndsi1.txt}
\bigskip

\end{frame}
%%%%%%%%%%%%%%%%%%%%%%%%%%%%%%%%%%%%%%%%%%%%%%%%%%%%%%%%%%%%%%%%%%%%%%%

\begin{frame}{Plot NDSI}
Qui ho usato la colorazione alternativa CIVIDIS (CVD friendly)
\bigskip

\scriptsize\lstinputlisting[language=R]{plot.txt} 

\end{frame}
%%%%%%%%%%%%%%%%%%%%%%%%%%%%%%%%%%%%%%%%%%%%%%%%%%%%%%%%%%%%%%%%%%%%%%%%
\begin{frame}{Plot NDSI}
\centering
\includegraphics[width=0.4\textwidth]{Rplot04.png}  
\includegraphics[width=0.4\textwidth]{Rplot05.png}  
\includegraphics[width=0.4\textwidth]{Rplot06.png}  
\includegraphics[width=0.4\textwidth]{Rplot07.png}  
    
\end{frame}


%%%%%%%%%%%%%%%%%%%%%%%%%%%%%%%%%%%%%%%%%%%%%%%%%%%%%%%%%%%%%%%%%%%%%%%%

\begin{frame}{Differenza NDSI}

A questo punto possiamo fare la differenza tra i 2 NDSI:
\bigskip

\scriptsize\lstinputlisting[language=R]{diff.txt} 
    
\end{frame}

%%%%%%%%%%%%%%%%%%%%%%%%%%%%%%%%%%%%%%%%%%%%%%%%%%%%%%%%%%%%%%%%%%%%%%%%
\begin{frame}{Plot differenza NDSI}
\centering
\includegraphics[width=0.45\textwidth]{Rplot08.png}  
\includegraphics[width=0.45\textwidth]{Rplot09.png}  
    
\end{frame}


%%%%%%%%%%%%%%%%%%%%%%%%%%%%%%%%%%%%%%%%%%%%%%%%%%%%%%%%%%%%%%%%%%%%%%%
%%%%%%%%%%%%%%%%%%%%%%%%%%%%%%%%%%%%%%%%%%%%%%%%%%%%%%%%%%%%%%%%%%%%%%%

\section{Previsioni future}
\bigskip
\begin{frame}{La situazione migliorerà o peggiorerà?}
A questa risposta non possiamo rispondere con estrema certezza, ma possiamo scoprire insieme i dati futuri per capire come potrebbe evolversi la storia del monte più alto del Giappone.
\end{frame}

%%%%%%%%%%%%%%%%%%%%%%%%%%%%%%%%%%%%%%%%%%%%%%%%%%%%%%%%%%%%%%%%%%%%%%%


\begin{frame}{Previsioni future}
Per fare questo abbiamo bisogno di dati passati per fare un paragone affidabile:
\bigskip
\scriptsize\lstinputlisting[language=R]{datifut.txt} 
    
\end{frame}
%%%%%%%%%%%%%%%%%%%%%%%%%%%%%%%%%%%%%%%%%%%%%%%%%%%%%%%%%%%%%%%%%%%%%%%%%

\begin{frame}{Dati 1970-2000}
Adesso possiamo usare GGplot per visualizzare il nostro risultato:
\tiny\lstinputlisting[language=R]{ggplot2.txt} 

\end{frame}

%%%%%%%%%%%%%%%%%%%%%%%%%%%%%%%%%%%%%%%%%%%%%%%%%%%%%%%%%%%%%%%%%%%%%%%%
\begin{frame}{Plot temperatura media 1970-2000}
    \centering
\includegraphics[width=0.6\textwidth]{Rplot11.png}  
\end{frame}

%%%%%%%%%%%%%%%%%%%%%%%%%%%%%%%%%%%%%%%%%%%%%%%%%%%%%%%%%%%%%%%%%%%%%%%%   
\begin{frame}{Dati futuri 2021-2040}
\bigskip

\tiny\lstinputlisting[language=R]{ggplot1.txt} 
    
\end{frame}
%%%%%%%%%%%%%%%%%%%%%%%%%%%%%%%%%%%%%%%%%%%%%%%%%%%%%%%%%%%%%%%%%%%%%%%%%
\begin{frame}{Plot temperatua massima 2021-2040}
    \centering
\includegraphics[width=0.6\textwidth]{Rplot10.png}  
\end{frame}

%%%%%%%%%%%%%%%%%%%%%%%%%%%%%%%%%%%%%%%%%%%%%%%%%%%%%%%%%%%%%%%%%%%%%%%%%
\begin{frame}{Vediamo i plot insieme}
\tiny\lstinputlisting[language=R]{arr.txt} 

\centering
\includegraphics[width=0.45\textwidth]{Rplot10.png}  
\includegraphics[width=0.45\textwidth]{Rplot11.png}  
\end{frame}
%%%%%%%%%%%%%%%%%%%%%%%%%%%%%%%%%%%%%%%%%%%%%%%%%%%%%%%%%%%%%%%%%%%%%%%%
\begin{frame}{Queste immagini sono comprensibili?}

Ma a questo punto, siamo in grado di comprendere visivamente e con precisione l'innalzamento della temperatura?

    
\end{frame}

%%%%%%%%%%%%%%%%%%%%%%%%%%%%%%%%%%%%%%%%%%%%%%%%%%%%%%%%%%%%%%%%%%%%%%%%%%

\begin{frame}{Istogrammi}
Ci vengono in aiuto gli istogrammi che riescono a darci una visione precisa dell'incremento stimato di temperatura entro il 2040.
    
\end{frame}


%%%%%%%%%%%%%%%%%%%%%%%%%%%%%%%%%%%%%%%%%%%%%%%%%%%%%%%%%%%%%%%%%%%%%%%%%%%%

\begin{frame}{Istogrammi}
\centering
    \tiny\lstinputlisting[language=R]{isto.txt} 
\end{frame}

%%%%%%%%%%%%%%%%%%%%%%%%%%%%%%%%%%%%%%%%%%%%%%%%%%%%%%%%%%%%%%%%%%%%%%%%%%%%%
\begin{frame}{Plot istogramma}
\centering
\includegraphics[width=1\textwidth]{Rplot13.png}  
    
\end{frame}

%%%%%%%%%%%%%%%%%%%%%%%%%%%%%%%%%%%%%%%%%%%%%%%%%%%%%%%%%%%%%%%%%%%%%%%%%%%%%

\begin{frame}
\centering
    GRAZIE DELL'ATTENZIONE!
    \includegraphics[width=1\textwidth]{intro.jpg}  
\end{frame}
%%%%%%%%%%%%%%%%%%%%%%%%%%%%%%%%%%%%%%%%%%%%%%%%%%%%%%%%%%%%%%%%%%%%%%%%%%%%%%%
\end{document}
